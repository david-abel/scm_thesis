%% preamble.tex
%% this should be included with a command like
%% %% preamble.tex
%% this should be included with a command like
%% %% preamble.tex
%% this should be included with a command like
%% %% preamble.tex
%% this should be included with a command like
%% \input{p}

\usepackage{amsfonts}
\usepackage{amsthm}
\usepackage{latexsym}
\usepackage{amsmath}
%\usepackage{newalg}
\usepackage{amssymb}
\usepackage{latexsym}
\usepackage{subfigure}
\usepackage{graphicx}
\usepackage{wasysym}
\usepackage{booktabs}
\usepackage{hyperref}
\usepackage{mathtools}
\usepackage[makeroom]{cancel}
\usepackage{enumerate}
\usepackage{framed}
\usepackage{tikz}
\usepackage{mdframed}
\usepackage{algorithm}
\usepackage[noend]{algpseudocode}

% THEOREMS AND DEFINITIONS
\newcommand{\sk}{s}
\newtheorem{theorem}{Theorem}
\newtheorem{lesson}{Lesson}
\newtheorem{proposition}{Proposition}
\newtheorem{lemma}{Lemma}
\newtheorem{corollary}{Corollary}
\newtheorem{fact}{Fact}
\newtheorem*{claim}{Claim}
%\theoremstyle{definition}
\newtheorem{definition}{Definition}
\newtheorem{assumption}{Assumption}
\theoremstyle{remark}
\newtheorem{example}{Example}
\newtheorem*{remark}{Remark}

% DEFINITION command
%\newcounter{definitionCount}
%\setcounter{definitionCount}{0}
%\newcommand{\definition}[1]{
%\addtocounter{definitionCount}{1}
%\begin{mdframed}[backgroundcolor=gray!3,roundcorner=4pt]\textbf{ Definition \arabic{definitionCount}:} #1
%\end{mdframed}}

%==============================================================================
% Macros.
%==============================================================================
\newcommand{\new}[1]{{\em #1\/}}		% New term (set in italics).
\newcommand{\set}[1]{\{#1\}}			% Set (as in \set{1,2,3})
\newcommand{\setof}[2]{\{\,{#1}|~{#2}\,\}}	% Set (as in \setof{x}{x > 0})
\newcommand{\C}{\mathbb{C}}	                % Complex numbers.
\newcommand{\Q}{\textsf{Q}}                     % Robinson Arithmetic
\newcommand{\R}{\textsf{R}}                     % The system R
\newcommand{\PA}{\textsf{PA}}                     % Peano Arithmetic
\newcommand{\LA}{$\mathcal{LA}$}			% Language of Arithmetic
\newcommand{\compl}[1]{\overline{#1}}		% Complement of ...     
\newcommand{\subproblem}[1]{\vspace{4 mm}\noindent {\bf (#1)} \\} % Subproblem (a), (b), etc...
\newcommand{\elem}[1]{\noindent{\bf #1}}	    % Proof element (Claim:, Proof:, etc.)
\newcommand{\spacerule}{\vspace{4mm}\hrule\vspace{4mm}} % Add an hrule with some space
\newcommand{\dmod}[2]{\left[ #1\ \text{mod}\ #2\right]} % My mod rule
\newcommand{\tab}[1]{\hspace{.2\textwidth}\rlap{#1}}

\newcommand{\ra}[1]{\renewcommand{\arraystretch}{#1}} % booktabs table cmd for spacing

% Misc lines and other formatting
\newcommand{\encircle}[1]{\tikz[baseline=(char.base)]\node[anchor=south west, draw,rectangle, rounded corners, inner sep=2pt, minimum size=6mm, text height=2mm](char){#1} ;} % Circled text
\newcommand{\midline}{
\begin{center}
\noindent\rule{4cm}{0.4pt}
\end{center}}

% GODEL NUMBERS
\newbox\gnBoxA
\newdimen\gnCornerHgt
\setbox\gnBoxA=\hbox{$\ulcorner$}
\global\gnCornerHgt=\ht\gnBoxA
\newdimen\gnArgHgt
\def\Godelnum #1{%
\setbox\gnBoxA=\hbox{$#1$}%
\gnArgHgt=\ht\gnBoxA%
\ifnum     \gnArgHgt<\gnCornerHgt \gnArgHgt=0pt%
\else \advance \gnArgHgt by -\gnCornerHgt%
\fi \raise\gnArgHgt\hbox{$\ulcorner$} \box\gnBoxA %
\raise\gnArgHgt\hbox{$\urcorner$}}

\DeclarePairedDelimiter{\ceil}{\lceil}{\rceil}
\DeclareSymbolFont{AMSb}{U}{msb}{m}{n}
\DeclareMathSymbol{\F}{\mathalpha}{AMSb}{"46}
\DeclareMathSymbol{\N}{\mathalpha}{AMSb}{"4E}
\DeclareMathSymbol{\X}{\mathalpha}{AMSb}{"58}
\DeclareMathSymbol{\Zz}{\mathalpha}{AMSb}{"5A}
\DeclareMathOperator*{\argmin}{arg\,min}
\DeclareMathOperator*{\argmax}{arg\,max}
\newcommand{\Z}[1]{{\ensuremath{\Zz_{#1}} }}
\newcommand{\Zs}[1]{\ensuremath{\Zz^{\ast}_{#1}}}
\newcommand{\Zn}{\Z{n}}
\newcommand{\Zns}{\Zs{n}}
\newcommand{\Zstar}[1]{\Zs{#1}}
\newcommand{\Zp}{{\Z{p}}}
\newcommand{\Zps}{\Zs{p}}
\newcommand{\Zqs}{\Zs{q}}
\newcommand{\Zq}{{\Z{q}}}
%\newcommand{\QR}{\mathop{\mathrm{QR}}\nolimits}
\newcommand{\ord}[1]{\mathop{\mathrm{ord}}({#1})}
\newcommand{\QR}[1]{\ensuremath{\textit{QR}_{#1}}}
\newcommand{\becomes}{:=}
\newcommand{\rem}[1]{\ensuremath{\ \operatorname{rem} #1}}  
\newcommand{\U}{{\mathcal{U}}}
\newcommand{\floor}[1]{\ensuremath{\lfloor{#1}\rfloor}}
\newcommand{\de}[1]{\ensuremath{\Delta{#1}}}
\newcommand{\js}[2]{\left( \frac{#1}{#2} \right)}
\newcommand{\E}[1]{{\bf E} \left[ #1 \right]}
\newcommand{\PR}[1]{{\bf Pr} \left[ #1 \right]}

\renewcommand{\QR}{{\mbox{QR}}}
\newcommand{\QNR}{{\mbox{QNR}}}
\newcommand{\crt}{{\mbox{CRT}}}
\newcommand{\rsa}{{\mbox{RSA}}}
\newcommand{\rsamod}{{\mbox{RSA-modulus}}}


\newcommand{\greq}[1]{\stackrel{#1}{=}}
\newcommand{\hash}{\ensuremath{\mathcal{H}}}
\newcommand{\negl}{{\tt neg}}
\newcommand{\cindist}{\stackrel{c}{\approx}}
\newcommand{\A}{{\mathcal{A}}}
\newcommand{\B}{{\mathcal{B}}}

\newcommand{\gen}{\mathit{Gen}}
\newcommand{\keygen}{\mathit{KeyGen}}
\newcommand{\RSA}{\mathit{RSA}}
\newcommand{\pk}{\mathit{pk}}
\newcommand{\PK}{\mathit{PK}}
\newcommand{\SK}{\mathit{SK}}
\newcommand{\sign}{\mathit{Sign}}
\newcommand{\verify}{\mathit{Verify}}

% Bew command
\newcommand{\Bew}[1]{\textsc{Bew}(\Godelnum{#1})}


\setlength{\oddsidemargin}{.25in}
\setlength{\evensidemargin}{.25in}
\setlength{\textwidth}{6in}
\setlength{\topmargin}{-0.4in}
\setlength{\textheight}{8.5in}

\newcommand{\handout}[5]{
   \renewcommand{\thepage}{#1-\arabic{page}}
   \noindent
   \begin{center}
   \framebox{
      \vbox{
    \hbox to 5.78in { {\bf PHIL1885: Incompleteness} \hfill #2 }
       \vspace{4mm}
       \hbox to 5.78in { {\Large \hfill #5  \hfill} }
       \vspace{2mm}
       \hbox to 5.78in { {\it #3 \hfill #4} }
      }
   }
   \end{center}
   \vspace*{4mm}
}

% REMOVES ALL INDENTATION
%\setlength{\parindent}{0pt}

\usepackage{parskip}

\newcommand{\ho}[4]{\handout{#1}{#2}{Instructor:
#3}{}{Handout #1: #4}}

\newcommand{\lnotes}[3]{\handout{#1}{#2}{Instructor:
#3}{}{Lecture #1}}

\newcommand{\solution}[1]{#1}
%\homework{number}{out}{due}{instructor}
\newcommand{\homework}[4]{\handout{HW #1}{#2}{Instructor: #4}{Due: #3}{Homework #1}}

%================================================
% problemset macros
%================================================
% count problems
%\newcounter{solutioncount}
%\setcounter{solutioncount}{0}
\newcommand{\problem}[1]{%
%\addtocounter{solutioncount}{1}%
\section*{Problem #1:}}

% lets you make alphabetical lists (at first-level of enumeration)
\newenvironment
  {alphabetize}{\renewcommand{\theenumi}{\alph{enumi}}\begin{enumerate}}
  {\end{enumerate}\renewcommand{\theenumi}{\arabic{enumi}}}

\usepackage{amsfonts}
\usepackage{amsthm}
\usepackage{latexsym}
\usepackage{amsmath}
%\usepackage{newalg}
\usepackage{amssymb}
\usepackage{latexsym}
\usepackage{subfigure}
\usepackage{graphicx}
\usepackage{wasysym}
\usepackage{booktabs}
\usepackage{hyperref}
\usepackage{mathtools}
\usepackage[makeroom]{cancel}
\usepackage{enumerate}
\usepackage{framed}
\usepackage{tikz}
\usepackage{mdframed}
\usepackage{algorithm}
\usepackage[noend]{algpseudocode}

% THEOREMS AND DEFINITIONS
\newcommand{\sk}{s}
\newtheorem{theorem}{Theorem}
\newtheorem{lesson}{Lesson}
\newtheorem{proposition}{Proposition}
\newtheorem{lemma}{Lemma}
\newtheorem{corollary}{Corollary}
\newtheorem{fact}{Fact}
\newtheorem*{claim}{Claim}
%\theoremstyle{definition}
\newtheorem{definition}{Definition}
\newtheorem{assumption}{Assumption}
\theoremstyle{remark}
\newtheorem{example}{Example}
\newtheorem*{remark}{Remark}

% DEFINITION command
%\newcounter{definitionCount}
%\setcounter{definitionCount}{0}
%\newcommand{\definition}[1]{
%\addtocounter{definitionCount}{1}
%\begin{mdframed}[backgroundcolor=gray!3,roundcorner=4pt]\textbf{ Definition \arabic{definitionCount}:} #1
%\end{mdframed}}

%==============================================================================
% Macros.
%==============================================================================
\newcommand{\new}[1]{{\em #1\/}}		% New term (set in italics).
\newcommand{\set}[1]{\{#1\}}			% Set (as in \set{1,2,3})
\newcommand{\setof}[2]{\{\,{#1}|~{#2}\,\}}	% Set (as in \setof{x}{x > 0})
\newcommand{\C}{\mathbb{C}}	                % Complex numbers.
\newcommand{\Q}{\textsf{Q}}                     % Robinson Arithmetic
\newcommand{\R}{\textsf{R}}                     % The system R
\newcommand{\PA}{\textsf{PA}}                     % Peano Arithmetic
\newcommand{\LA}{$\mathcal{LA}$}			% Language of Arithmetic
\newcommand{\compl}[1]{\overline{#1}}		% Complement of ...     
\newcommand{\subproblem}[1]{\vspace{4 mm}\noindent {\bf (#1)} \\} % Subproblem (a), (b), etc...
\newcommand{\elem}[1]{\noindent{\bf #1}}	    % Proof element (Claim:, Proof:, etc.)
\newcommand{\spacerule}{\vspace{4mm}\hrule\vspace{4mm}} % Add an hrule with some space
\newcommand{\dmod}[2]{\left[ #1\ \text{mod}\ #2\right]} % My mod rule
\newcommand{\tab}[1]{\hspace{.2\textwidth}\rlap{#1}}

\newcommand{\ra}[1]{\renewcommand{\arraystretch}{#1}} % booktabs table cmd for spacing

% Misc lines and other formatting
\newcommand{\encircle}[1]{\tikz[baseline=(char.base)]\node[anchor=south west, draw,rectangle, rounded corners, inner sep=2pt, minimum size=6mm, text height=2mm](char){#1} ;} % Circled text
\newcommand{\midline}{
\begin{center}
\noindent\rule{4cm}{0.4pt}
\end{center}}

% GODEL NUMBERS
\newbox\gnBoxA
\newdimen\gnCornerHgt
\setbox\gnBoxA=\hbox{$\ulcorner$}
\global\gnCornerHgt=\ht\gnBoxA
\newdimen\gnArgHgt
\def\Godelnum #1{%
\setbox\gnBoxA=\hbox{$#1$}%
\gnArgHgt=\ht\gnBoxA%
\ifnum     \gnArgHgt<\gnCornerHgt \gnArgHgt=0pt%
\else \advance \gnArgHgt by -\gnCornerHgt%
\fi \raise\gnArgHgt\hbox{$\ulcorner$} \box\gnBoxA %
\raise\gnArgHgt\hbox{$\urcorner$}}

\DeclarePairedDelimiter{\ceil}{\lceil}{\rceil}
\DeclareSymbolFont{AMSb}{U}{msb}{m}{n}
\DeclareMathSymbol{\F}{\mathalpha}{AMSb}{"46}
\DeclareMathSymbol{\N}{\mathalpha}{AMSb}{"4E}
\DeclareMathSymbol{\X}{\mathalpha}{AMSb}{"58}
\DeclareMathSymbol{\Zz}{\mathalpha}{AMSb}{"5A}
\DeclareMathOperator*{\argmin}{arg\,min}
\DeclareMathOperator*{\argmax}{arg\,max}
\newcommand{\Z}[1]{{\ensuremath{\Zz_{#1}} }}
\newcommand{\Zs}[1]{\ensuremath{\Zz^{\ast}_{#1}}}
\newcommand{\Zn}{\Z{n}}
\newcommand{\Zns}{\Zs{n}}
\newcommand{\Zstar}[1]{\Zs{#1}}
\newcommand{\Zp}{{\Z{p}}}
\newcommand{\Zps}{\Zs{p}}
\newcommand{\Zqs}{\Zs{q}}
\newcommand{\Zq}{{\Z{q}}}
%\newcommand{\QR}{\mathop{\mathrm{QR}}\nolimits}
\newcommand{\ord}[1]{\mathop{\mathrm{ord}}({#1})}
\newcommand{\QR}[1]{\ensuremath{\textit{QR}_{#1}}}
\newcommand{\becomes}{:=}
\newcommand{\rem}[1]{\ensuremath{\ \operatorname{rem} #1}}  
\newcommand{\U}{{\mathcal{U}}}
\newcommand{\floor}[1]{\ensuremath{\lfloor{#1}\rfloor}}
\newcommand{\de}[1]{\ensuremath{\Delta{#1}}}
\newcommand{\js}[2]{\left( \frac{#1}{#2} \right)}
\newcommand{\E}[1]{{\bf E} \left[ #1 \right]}
\newcommand{\PR}[1]{{\bf Pr} \left[ #1 \right]}

\renewcommand{\QR}{{\mbox{QR}}}
\newcommand{\QNR}{{\mbox{QNR}}}
\newcommand{\crt}{{\mbox{CRT}}}
\newcommand{\rsa}{{\mbox{RSA}}}
\newcommand{\rsamod}{{\mbox{RSA-modulus}}}


\newcommand{\greq}[1]{\stackrel{#1}{=}}
\newcommand{\hash}{\ensuremath{\mathcal{H}}}
\newcommand{\negl}{{\tt neg}}
\newcommand{\cindist}{\stackrel{c}{\approx}}
\newcommand{\A}{{\mathcal{A}}}
\newcommand{\B}{{\mathcal{B}}}

\newcommand{\gen}{\mathit{Gen}}
\newcommand{\keygen}{\mathit{KeyGen}}
\newcommand{\RSA}{\mathit{RSA}}
\newcommand{\pk}{\mathit{pk}}
\newcommand{\PK}{\mathit{PK}}
\newcommand{\SK}{\mathit{SK}}
\newcommand{\sign}{\mathit{Sign}}
\newcommand{\verify}{\mathit{Verify}}

% Bew command
\newcommand{\Bew}[1]{\textsc{Bew}(\Godelnum{#1})}


\setlength{\oddsidemargin}{.25in}
\setlength{\evensidemargin}{.25in}
\setlength{\textwidth}{6in}
\setlength{\topmargin}{-0.4in}
\setlength{\textheight}{8.5in}

\newcommand{\handout}[5]{
   \renewcommand{\thepage}{#1-\arabic{page}}
   \noindent
   \begin{center}
   \framebox{
      \vbox{
    \hbox to 5.78in { {\bf PHIL1885: Incompleteness} \hfill #2 }
       \vspace{4mm}
       \hbox to 5.78in { {\Large \hfill #5  \hfill} }
       \vspace{2mm}
       \hbox to 5.78in { {\it #3 \hfill #4} }
      }
   }
   \end{center}
   \vspace*{4mm}
}

% REMOVES ALL INDENTATION
%\setlength{\parindent}{0pt}

\usepackage{parskip}

\newcommand{\ho}[4]{\handout{#1}{#2}{Instructor:
#3}{}{Handout #1: #4}}

\newcommand{\lnotes}[3]{\handout{#1}{#2}{Instructor:
#3}{}{Lecture #1}}

\newcommand{\solution}[1]{#1}
%\homework{number}{out}{due}{instructor}
\newcommand{\homework}[4]{\handout{HW #1}{#2}{Instructor: #4}{Due: #3}{Homework #1}}

%================================================
% problemset macros
%================================================
% count problems
%\newcounter{solutioncount}
%\setcounter{solutioncount}{0}
\newcommand{\problem}[1]{%
%\addtocounter{solutioncount}{1}%
\section*{Problem #1:}}

% lets you make alphabetical lists (at first-level of enumeration)
\newenvironment
  {alphabetize}{\renewcommand{\theenumi}{\alph{enumi}}\begin{enumerate}}
  {\end{enumerate}\renewcommand{\theenumi}{\arabic{enumi}}}

\usepackage{amsfonts}
\usepackage{amsthm}
\usepackage{latexsym}
\usepackage{amsmath}
%\usepackage{newalg}
\usepackage{amssymb}
\usepackage{latexsym}
\usepackage{subfigure}
\usepackage{graphicx}
\usepackage{wasysym}
\usepackage{booktabs}
\usepackage{hyperref}
\usepackage{mathtools}
\usepackage[makeroom]{cancel}
\usepackage{enumerate}
\usepackage{framed}
\usepackage{tikz}
\usepackage{mdframed}
\usepackage{algorithm}
\usepackage[noend]{algpseudocode}

% THEOREMS AND DEFINITIONS
\newcommand{\sk}{s}
\newtheorem{theorem}{Theorem}
\newtheorem{lesson}{Lesson}
\newtheorem{proposition}{Proposition}
\newtheorem{lemma}{Lemma}
\newtheorem{corollary}{Corollary}
\newtheorem{fact}{Fact}
\newtheorem*{claim}{Claim}
%\theoremstyle{definition}
\newtheorem{definition}{Definition}
\newtheorem{assumption}{Assumption}
\theoremstyle{remark}
\newtheorem{example}{Example}
\newtheorem*{remark}{Remark}

% DEFINITION command
%\newcounter{definitionCount}
%\setcounter{definitionCount}{0}
%\newcommand{\definition}[1]{
%\addtocounter{definitionCount}{1}
%\begin{mdframed}[backgroundcolor=gray!3,roundcorner=4pt]\textbf{ Definition \arabic{definitionCount}:} #1
%\end{mdframed}}

%==============================================================================
% Macros.
%==============================================================================
\newcommand{\new}[1]{{\em #1\/}}		% New term (set in italics).
\newcommand{\set}[1]{\{#1\}}			% Set (as in \set{1,2,3})
\newcommand{\setof}[2]{\{\,{#1}|~{#2}\,\}}	% Set (as in \setof{x}{x > 0})
\newcommand{\C}{\mathbb{C}}	                % Complex numbers.
\newcommand{\Q}{\textsf{Q}}                     % Robinson Arithmetic
\newcommand{\R}{\textsf{R}}                     % The system R
\newcommand{\PA}{\textsf{PA}}                     % Peano Arithmetic
\newcommand{\LA}{$\mathcal{LA}$}			% Language of Arithmetic
\newcommand{\compl}[1]{\overline{#1}}		% Complement of ...     
\newcommand{\subproblem}[1]{\vspace{4 mm}\noindent {\bf (#1)} \\} % Subproblem (a), (b), etc...
\newcommand{\elem}[1]{\noindent{\bf #1}}	    % Proof element (Claim:, Proof:, etc.)
\newcommand{\spacerule}{\vspace{4mm}\hrule\vspace{4mm}} % Add an hrule with some space
\newcommand{\dmod}[2]{\left[ #1\ \text{mod}\ #2\right]} % My mod rule
\newcommand{\tab}[1]{\hspace{.2\textwidth}\rlap{#1}}

\newcommand{\ra}[1]{\renewcommand{\arraystretch}{#1}} % booktabs table cmd for spacing

% Misc lines and other formatting
\newcommand{\encircle}[1]{\tikz[baseline=(char.base)]\node[anchor=south west, draw,rectangle, rounded corners, inner sep=2pt, minimum size=6mm, text height=2mm](char){#1} ;} % Circled text
\newcommand{\midline}{
\begin{center}
\noindent\rule{4cm}{0.4pt}
\end{center}}

% GODEL NUMBERS
\newbox\gnBoxA
\newdimen\gnCornerHgt
\setbox\gnBoxA=\hbox{$\ulcorner$}
\global\gnCornerHgt=\ht\gnBoxA
\newdimen\gnArgHgt
\def\Godelnum #1{%
\setbox\gnBoxA=\hbox{$#1$}%
\gnArgHgt=\ht\gnBoxA%
\ifnum     \gnArgHgt<\gnCornerHgt \gnArgHgt=0pt%
\else \advance \gnArgHgt by -\gnCornerHgt%
\fi \raise\gnArgHgt\hbox{$\ulcorner$} \box\gnBoxA %
\raise\gnArgHgt\hbox{$\urcorner$}}

\DeclarePairedDelimiter{\ceil}{\lceil}{\rceil}
\DeclareSymbolFont{AMSb}{U}{msb}{m}{n}
\DeclareMathSymbol{\F}{\mathalpha}{AMSb}{"46}
\DeclareMathSymbol{\N}{\mathalpha}{AMSb}{"4E}
\DeclareMathSymbol{\X}{\mathalpha}{AMSb}{"58}
\DeclareMathSymbol{\Zz}{\mathalpha}{AMSb}{"5A}
\DeclareMathOperator*{\argmin}{arg\,min}
\DeclareMathOperator*{\argmax}{arg\,max}
\newcommand{\Z}[1]{{\ensuremath{\Zz_{#1}} }}
\newcommand{\Zs}[1]{\ensuremath{\Zz^{\ast}_{#1}}}
\newcommand{\Zn}{\Z{n}}
\newcommand{\Zns}{\Zs{n}}
\newcommand{\Zstar}[1]{\Zs{#1}}
\newcommand{\Zp}{{\Z{p}}}
\newcommand{\Zps}{\Zs{p}}
\newcommand{\Zqs}{\Zs{q}}
\newcommand{\Zq}{{\Z{q}}}
%\newcommand{\QR}{\mathop{\mathrm{QR}}\nolimits}
\newcommand{\ord}[1]{\mathop{\mathrm{ord}}({#1})}
\newcommand{\QR}[1]{\ensuremath{\textit{QR}_{#1}}}
\newcommand{\becomes}{:=}
\newcommand{\rem}[1]{\ensuremath{\ \operatorname{rem} #1}}  
\newcommand{\U}{{\mathcal{U}}}
\newcommand{\floor}[1]{\ensuremath{\lfloor{#1}\rfloor}}
\newcommand{\de}[1]{\ensuremath{\Delta{#1}}}
\newcommand{\js}[2]{\left( \frac{#1}{#2} \right)}
\newcommand{\E}[1]{{\bf E} \left[ #1 \right]}
\newcommand{\PR}[1]{{\bf Pr} \left[ #1 \right]}

\renewcommand{\QR}{{\mbox{QR}}}
\newcommand{\QNR}{{\mbox{QNR}}}
\newcommand{\crt}{{\mbox{CRT}}}
\newcommand{\rsa}{{\mbox{RSA}}}
\newcommand{\rsamod}{{\mbox{RSA-modulus}}}


\newcommand{\greq}[1]{\stackrel{#1}{=}}
\newcommand{\hash}{\ensuremath{\mathcal{H}}}
\newcommand{\negl}{{\tt neg}}
\newcommand{\cindist}{\stackrel{c}{\approx}}
\newcommand{\A}{{\mathcal{A}}}
\newcommand{\B}{{\mathcal{B}}}

\newcommand{\gen}{\mathit{Gen}}
\newcommand{\keygen}{\mathit{KeyGen}}
\newcommand{\RSA}{\mathit{RSA}}
\newcommand{\pk}{\mathit{pk}}
\newcommand{\PK}{\mathit{PK}}
\newcommand{\SK}{\mathit{SK}}
\newcommand{\sign}{\mathit{Sign}}
\newcommand{\verify}{\mathit{Verify}}

% Bew command
\newcommand{\Bew}[1]{\textsc{Bew}(\Godelnum{#1})}


\setlength{\oddsidemargin}{.25in}
\setlength{\evensidemargin}{.25in}
\setlength{\textwidth}{6in}
\setlength{\topmargin}{-0.4in}
\setlength{\textheight}{8.5in}

\newcommand{\handout}[5]{
   \renewcommand{\thepage}{#1-\arabic{page}}
   \noindent
   \begin{center}
   \framebox{
      \vbox{
    \hbox to 5.78in { {\bf PHIL1885: Incompleteness} \hfill #2 }
       \vspace{4mm}
       \hbox to 5.78in { {\Large \hfill #5  \hfill} }
       \vspace{2mm}
       \hbox to 5.78in { {\it #3 \hfill #4} }
      }
   }
   \end{center}
   \vspace*{4mm}
}

% REMOVES ALL INDENTATION
%\setlength{\parindent}{0pt}

\usepackage{parskip}

\newcommand{\ho}[4]{\handout{#1}{#2}{Instructor:
#3}{}{Handout #1: #4}}

\newcommand{\lnotes}[3]{\handout{#1}{#2}{Instructor:
#3}{}{Lecture #1}}

\newcommand{\solution}[1]{#1}
%\homework{number}{out}{due}{instructor}
\newcommand{\homework}[4]{\handout{HW #1}{#2}{Instructor: #4}{Due: #3}{Homework #1}}

%================================================
% problemset macros
%================================================
% count problems
%\newcounter{solutioncount}
%\setcounter{solutioncount}{0}
\newcommand{\problem}[1]{%
%\addtocounter{solutioncount}{1}%
\section*{Problem #1:}}

% lets you make alphabetical lists (at first-level of enumeration)
\newenvironment
  {alphabetize}{\renewcommand{\theenumi}{\alph{enumi}}\begin{enumerate}}
  {\end{enumerate}\renewcommand{\theenumi}{\arabic{enumi}}}

\usepackage{amsfonts}
\usepackage{amsthm}
\usepackage{latexsym}
\usepackage{amsmath}
%\usepackage{newalg}
\usepackage{amssymb}
\usepackage{latexsym}
\usepackage{graphicx}
\usepackage{wasysym}
\usepackage{hyperref}
\usepackage{mathtools}
\usepackage[makeroom]{cancel}
\usepackage{enumerate}
\usepackage{framed}
\usepackage{tikz}
\usepackage{mdframed}
\usepackage{algorithm}
\usepackage[noend]{algpseudocode}

% THEOREMS AND DEFINITIONS
\newcommand{\sk}{s}
\newtheorem{theorem}{Theorem}
\newtheorem{lesson}{Lesson}
\newtheorem{proposition}{Proposition}
\newtheorem{lemma}{Lemma}
\newtheorem{corollary}{Corollary}
\newtheorem{fact}{Fact}
\newtheorem*{claim}{Claim}
%\theoremstyle{definition}
\newtheorem{definition}{Definition}
\newtheorem{assumption}{Assumption}
\theoremstyle{remark}
\newtheorem{example}{Example}
\newtheorem*{remark}{Remark}

% DEFINITION command
%\newcounter{definitionCount}
%\setcounter{definitionCount}{0}
%\newcommand{\definition}[1]{
%\addtocounter{definitionCount}{1}
%\begin{mdframed}[backgroundcolor=gray!3,roundcorner=4pt]\textbf{ Definition \arabic{definitionCount}:} #1
%\end{mdframed}}

%==============================================================================
% Macros.
%==============================================================================
\newcommand{\new}[1]{{\em #1\/}}		% New term (set in italics).
\newcommand{\set}[1]{\{#1\}}			% Set (as in \set{1,2,3})
\newcommand{\setof}[2]{\{\,{#1}|~{#2}\,\}}	% Set (as in \setof{x}{x > 0})
\newcommand{\C}{\mathbb{C}}	                % Complex numbers.
\newcommand{\Q}{\textsf{Q}}                     % Robinson Arithmetic
\newcommand{\R}{\textsf{R}}                     % The system R
\newcommand{\PA}{\textsf{PA}}                     % Peano Arithmetic
\newcommand{\LA}{$\mathcal{LA}$}			% Language of Arithmetic
\newcommand{\compl}[1]{\overline{#1}}		% Complement of ...     
\newcommand{\subproblem}[1]{\vspace{4 mm}\noindent {\bf (#1)} \\} % Subproblem (a), (b), etc...
\newcommand{\elem}[1]{\noindent{\bf #1}}	    % Proof element (Claim:, Proof:, etc.)
\newcommand{\spacerule}{\vspace{4mm}\hrule\vspace{4mm}} % Add an hrule with some space
\newcommand{\dmod}[2]{\left[ #1\ \text{mod}\ #2\right]} % My mod rule
\newcommand{\tab}[1]{\hspace{.2\textwidth}\rlap{#1}}



% Misc lines and other formatting
\newcommand{\encircle}[1]{\tikz[baseline=(char.base)]\node[anchor=south west, draw,rectangle, rounded corners, inner sep=2pt, minimum size=6mm, text height=2mm](char){#1} ;} % Circled text
\newcommand{\midline}{
\begin{center}
\noindent\rule{4cm}{0.4pt}
\end{center}}

% GODEL NUMBERS
\newbox\gnBoxA
\newdimen\gnCornerHgt
\setbox\gnBoxA=\hbox{$\ulcorner$}
\global\gnCornerHgt=\ht\gnBoxA
\newdimen\gnArgHgt
\def\Godelnum #1{%
\setbox\gnBoxA=\hbox{$#1$}%
\gnArgHgt=\ht\gnBoxA%
\ifnum     \gnArgHgt<\gnCornerHgt \gnArgHgt=0pt%
\else \advance \gnArgHgt by -\gnCornerHgt%
\fi \raise\gnArgHgt\hbox{$\ulcorner$} \box\gnBoxA %
\raise\gnArgHgt\hbox{$\urcorner$}}

\DeclarePairedDelimiter{\ceil}{\lceil}{\rceil}
\DeclareSymbolFont{AMSb}{U}{msb}{m}{n}
\DeclareMathSymbol{\F}{\mathalpha}{AMSb}{"46}
\DeclareMathSymbol{\N}{\mathalpha}{AMSb}{"4E}
\DeclareMathSymbol{\X}{\mathalpha}{AMSb}{"58}
\DeclareMathSymbol{\Zz}{\mathalpha}{AMSb}{"5A}
\DeclareMathOperator*{\argmin}{arg\,min}
\DeclareMathOperator*{\argmax}{arg\,max}
\newcommand{\Z}[1]{{\ensuremath{\Zz_{#1}} }}
\newcommand{\Zs}[1]{\ensuremath{\Zz^{\ast}_{#1}}}
\newcommand{\Zn}{\Z{n}}
\newcommand{\Zns}{\Zs{n}}
\newcommand{\Zstar}[1]{\Zs{#1}}
\newcommand{\Zp}{{\Z{p}}}
\newcommand{\Zps}{\Zs{p}}
\newcommand{\Zqs}{\Zs{q}}
\newcommand{\Zq}{{\Z{q}}}
%\newcommand{\QR}{\mathop{\mathrm{QR}}\nolimits}
\newcommand{\ord}[1]{\mathop{\mathrm{ord}}({#1})}
\newcommand{\QR}[1]{\ensuremath{\textit{QR}_{#1}}}
\newcommand{\becomes}{:=}
\newcommand{\rem}[1]{\ensuremath{\ \operatorname{rem} #1}}  
\newcommand{\U}{{\mathcal{U}}}
\newcommand{\floor}[1]{\ensuremath{\lfloor{#1}\rfloor}}
\newcommand{\de}[1]{\ensuremath{\Delta{#1}}}
\newcommand{\js}[2]{\left( \frac{#1}{#2} \right)}
\newcommand{\E}[1]{{\bf E} \left[ #1 \right]}
\newcommand{\PR}[1]{{\bf Pr} \left[ #1 \right]}

\renewcommand{\QR}{{\mbox{QR}}}
\newcommand{\QNR}{{\mbox{QNR}}}
\newcommand{\crt}{{\mbox{CRT}}}
\newcommand{\rsa}{{\mbox{RSA}}}
\newcommand{\rsamod}{{\mbox{RSA-modulus}}}


\newcommand{\greq}[1]{\stackrel{#1}{=}}
\newcommand{\hash}{\ensuremath{\mathcal{H}}}
\newcommand{\negl}{{\tt neg}}
\newcommand{\cindist}{\stackrel{c}{\approx}}
\newcommand{\A}{{\mathcal{A}}}
\newcommand{\B}{{\mathcal{B}}}

\newcommand{\gen}{\mathit{Gen}}
\newcommand{\keygen}{\mathit{KeyGen}}
\newcommand{\RSA}{\mathit{RSA}}
\newcommand{\pk}{\mathit{pk}}
\newcommand{\PK}{\mathit{PK}}
\newcommand{\SK}{\mathit{SK}}
\newcommand{\sign}{\mathit{Sign}}
\newcommand{\verify}{\mathit{Verify}}

% Bew command
\newcommand{\Bew}[1]{\textsc{Bew}(\Godelnum{#1})}


\setlength{\oddsidemargin}{.25in}
\setlength{\evensidemargin}{.25in}
\setlength{\textwidth}{6in}
\setlength{\topmargin}{-0.4in}
\setlength{\textheight}{8.5in}

\newcommand{\handout}[5]{
   \renewcommand{\thepage}{#1-\arabic{page}}
   \noindent
   \begin{center}
   \framebox{
      \vbox{
    \hbox to 5.78in { {\bf PHIL1885: Incompleteness} \hfill #2 }
       \vspace{4mm}
       \hbox to 5.78in { {\Large \hfill #5  \hfill} }
       \vspace{2mm}
       \hbox to 5.78in { {\it #3 \hfill #4} }
      }
   }
   \end{center}
   \vspace*{4mm}
}

% REMOVES ALL INDENTATION
%\setlength{\parindent}{0pt}

\usepackage{parskip}

\newcommand{\ho}[4]{\handout{#1}{#2}{Instructor:
#3}{}{Handout #1: #4}}

\newcommand{\lnotes}[3]{\handout{#1}{#2}{Instructor:
#3}{}{Lecture #1}}

\newcommand{\solution}[1]{#1}
%\homework{number}{out}{due}{instructor}
\newcommand{\homework}[4]{\handout{HW #1}{#2}{Instructor: #4}{Due: #3}{Homework #1}}

%================================================
% problemset macros
%================================================
% count problems
%\newcounter{solutioncount}
%\setcounter{solutioncount}{0}
\newcommand{\problem}[1]{%
%\addtocounter{solutioncount}{1}%
\section*{Problem #1:}}

% lets you make alphabetical lists (at first-level of enumeration)
\newenvironment
  {alphabetize}{\renewcommand{\theenumi}{\alph{enumi}}\begin{enumerate}}
  {\end{enumerate}\renewcommand{\theenumi}{\arabic{enumi}}}